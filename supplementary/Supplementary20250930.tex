\documentclass[12pt]{article}

% 기본 패키지
\usepackage[utf8]{inputenc}
\usepackage{graphicx}
\usepackage{amsmath, amssymb}
\usepackage{authblk}         % 저자 및 소속
\usepackage{natbib}          % 참고문헌 스타일
\usepackage{geometry}        % 페이지 여백 조정
\usepackage{setspace}        % 줄간격 조정
\usepackage{hyperref}        % 하이퍼링크
\usepackage{caption}
\usepackage{float}
\usepackage{color}
\usepackage{lineno}          % 줄 번호 (선택사항)
\usepackage{booktabs}
\renewcommand{\thetable}{S\arabic{table}}   % Supplementary Table 번호 설정
\renewcommand{\thefigure}{S\arabic{figure}} % Supplementary Figure 번호 설정

% 페이지 설정
\geometry{margin=1in}
\setstretch{1.5}             % 줄간격 1.5배

% 참고문헌 스타일
\bibliographystyle{unsrtnat} % 또는 naturemag


\begin{document}

\section*{Supplementary Overview}
Supplementary materials include:
\begin{itemize}
  \item Table S1: Top 100 genes ranked by FDR significance
  \item Table S2: Pathway enrichment across ADNI and ROSMAP
  \item Figure S1: GAT-derived gene interaction heatmap
  \item Figure S2: ROC curves comparing baseline models
  \item Figure S3: Uniform Manifold Approximation and Projection (UMAP) projection of MOVE embeddings
  \item Supplementary Discussion: Functional interpretation and experimental validation strategies
\end{itemize}

\section*{Notation}

Throughout this study, we use the following notation:

\begin{itemize}
  \item \( v_i \): Node representing gene \( i \) in the biological graph
  \item \( h_i \): Feature vector of gene \( i \)
  \item \( e_{ij} \): Raw attention score between genes \( i \) and \( j \)
  \item \( \alpha_{ij} \): Normalized attention weight between genes \( i \) and \( j \)
  \item \( \mathcal{N}_i \): Neighborhood of node \( v_i \) in the graph
  \item \( W \): Learnable weight matrix in GAT layer
  \item \( \vec{a} \): Attention coefficient vector in GAT
  \item \( \parallel \): Concatenation operator
  \item \( K \): Number of attention heads
  \item \( \mathbf{h}_i' \): Final GAT-derived embedding of gene \( i \)
  \item \( \mathbf{x}^{(m)} \): Input feature vector from omics modality \( m \)
  \item \( M \): Total number of omics modalities
  \item \( \mathbf{z} \in \mathbb{R}^l \): Latent representation inferred by MOVE
  \item \( q_\phi^{(m)}(\mathbf{z}|\mathbf{x}^{(m)}) \): Variational encoder for modality \( m \)
  \item \( p_\theta^{(m)}(\mathbf{x}^{(m)}|\mathbf{z}) \): Decoder for modality \( m \)
  \item \( \beta \): Weight for KL divergence in MOVE loss
  \item \( \lambda \): Weight for cross-modal regularization term
  \item \( \mathcal{L}_{\text{MOVE}} \): Variational autoencoder loss for multi-omics embedding
  \item \( \mathcal{L}_{\text{cross}} \): Cross-modal coherence regularization term
  \item \( \hat{\beta} \): Estimated regression coefficients from ElasticNet
  \item \( \alpha \): Mixing parameter between L1 and L2 penalties in ElasticNet
  \item \( \lambda \): Regularization strength in ElasticNet
  \item \( q(p_i) \): Adjusted p-value for feature \( i \) using Storey's FDR
  \item \( \pi_0 \): Estimated proportion of true null hypotheses
  \item \( m \): Total number of hypotheses tested
\end{itemize}

\section*{Supplementary Methods}

\subsection*{Graph Attention Networks (GAT)}

Each gene is modeled as node \( v_i \) with feature vector \( h_i \). Attention score:
\[
e_{ij} = \text{LeakyReLU}\left(\vec{a}^T \left[W h_i \parallel W h_j\right]\right)
\]
Normalized weight:
\[
\alpha_{ij} = \frac{\exp(e_{ij})}{\sum_{k \in \mathcal{N}_i} \exp(e_{ik})}
\]
Final representation:
\[
h_i^{'} = \parallel_{k=1}^K \sum_{j \in \mathcal{N}_i} \alpha_{ij}^{(k)} W^{(k)} h_j
\]
This formulation follows the original Graph Attention Network architecture \citep{velic}, and its limitations in biological graphs are discussed in \citep{xu2020gcn_limitations}. These refined embeddings \( \mathbf{h}_i' \) are then aggregated across omics modalities \( m = 1, \dots, M \), forming input \( \mathbf{x}^{(m)} \) to the MOVE module. 

\subsection*{MOVE: Multi-Omics Variational Embedding}

The latent representation \( \mathbf{z} \in \mathbb{R}^l \) is inferred via a modality-specific variational encoder-decoder architecture. The objective function is defined as:
\[
\mathcal{L}_{\text{MOVE}} = \sum_{m=1}^{M} \mathbb{E}_{q_\phi^{(m)}(\mathbf{z}|\mathbf{x}^{(m)})}[\log p_\theta^{(m)}(\mathbf{x}^{(m)}|\mathbf{z})] - \beta \cdot D_{\mathrm{KL}}(q_\phi^{(m)}(\mathbf{z}|\mathbf{x}^{(m)}) || p(\mathbf{z}))
\]
To ensure cross-modal coherence, a regularization term is added:
\[
\mathcal{L}_{\text{total}} = \mathcal{L}_{\text{MOVE}} + \lambda \cdot \mathcal{L}_{\text{cross}}
\]
This formulation is adapted from the original variational autoencoder framework \citep{kingma2014vae}, with multi-omics extensions inspired by \citep{wang2021move, allesoe2023move}.

To assess the biological fidelity of the learned latent space, we performed dimensionality reduction using UMAP on the inferred embeddings \( \mathbf{z} \). As shown in Supplementary Figure S3, samples from AD and control cohorts formed distinct clusters, suggesting that MOVE effectively captures disease-relevant molecular signatures. Furthermore, cluster-specific enrichment analysis revealed that latent dimensions correlate with known biological pathways, including neuroinflammation and tau pathology \citep{mcinnes2018umap, iturria2018multi}.

This embedding strategy enables interpretable abstraction of multi-omicsdata while preserving biological structure, facilitating downstream tasks such as classification, clustering, and biomarker prioritization.
\subsection*{ElasticNet Regression}
\[
\hat{\beta} = \arg \min_{\beta} \left\{ \frac{1}{2n} \| y - X\beta \|_2^2 + \lambda \left[\alpha \|\beta\|_1 + (1 - \alpha) \|\beta\|_2^2\right] \right\}
\]
ElasticNet combines L1 and L2 penalties for robust feature selection in high-dimensional settings \citep{zou2005regularization}.

\subsection*{Storey's FDR}
\[
q(p_i) = \inf_{t \geq p_i} \left\{ \frac{\pi_0 t}{|\{p_j \leq t\}| / m} \right\}
\]
False discovery rate control is based on Storey's direct approach \citep{storey2002fdr, storey2003statistical}, with theoretical foundations from \citep{benjamini1995controlling, benjamini2001control, dudoit2003multiple}. 

\subsection*{Unified Framework and Notational Integration}

This integrated framework—\( \mathbf{h}_i \rightarrow \mathbf{h}_i' \rightarrow \mathbf{x}^{(m)} \rightarrow \mathbf{z} \rightarrow \hat{\boldsymbol{\beta}} \rightarrow q(p_i) \)—enables biologically contextualized representation learning, modality-aware embedding, sparse predictive modeling, and rigorous statistical inference. The synergy among these components enhances interpretability, generalizability, and reproducibility in multi-omics biomarker discovery.

\section*{Supplementary Tables}
Table S1 in supplementary materials shows top 100 genes ranked by statistical significance.

\begin{table}[H]
\centering
\scriptsize
\begin{tabular}{ll|ll|ll|ll|ll}
\toprule
\textbf{Rank} & \textbf{Gene} & \textbf{Rank} & \textbf{Gene} & \textbf{Rank} & \textbf{Gene} & \textbf{Rank} & \textbf{Gene} & \textbf{Rank} & \textbf{Gene} \\
\midrule
1 & TREM2 & 21 & TYROBP & 41 & HLA-B & 61 & SLC24A4 & 81 & CD2AP \\
2 & APOE & 22 & CLU & 42 & CASS4 & 62 & CST3 & 82 & FERMT2 \\
3 & MAPT & 23 & BIN1 & 43 & SPI1 & 63 & ITGAX & 83 & ABCA7 \\
4 & PSEN1 & 24 & GRN & 44 & MS4A6A & 64 & PICALM & 84 & CD33 \\
5 & SORL1 & 25 & APP & 45 & NCSTN & 65 & APOC1 & 85 & HLA-A \\
6 & PLCG2 & 26 & CR1 & 46 & PTK2B & 66 & INPP5D & 86 & C1QA \\
7 & BACE1 & 27 & HLA-DRA & 47 & CST7 & 67 & SORBS1 & 87 & NME1 \\
8 & CASS4 & 28 & HLA-DRB1 & 48 & LILRB2 & 68 & CD74 & 88 & GSN \\
9 & CD33 & 29 & HLA-DQB1 & 49 & FCER1G & 69 & TLR2 & 89 & HSPA1A \\
10 & GRN & 30 & HLA-DPA1 & 50 & CTSD & 70 & S100A9 & 90 & HSPB1 \\
11 & CLU & 31 & HLA-DPB1 & 51 & CTSB & 71 & S100A8 & 91 & VIM \\
12 & BIN1 & 32 & HLA-C & 52 & CTSL & 72 & LGALS3 & 92 & ANXA1 \\
13 & PICALM & 33 & HLA-E & 53 & CTSK & 73 & CD68 & 93 & ANXA2 \\
14 & MS4A6A & 34 & HLA-F & 54 & CTSZ & 74 & CD14 & 94 & ACTB \\
15 & SPI1 & 35 & HLA-G & 55 & CTSO & 75 & CD86 & 95 & ACTG1 \\
16 & INPP5D & 36 & HLA-H & 56 & CTSV & 76 & CD80 & 96 & GAPDH \\
17 & ITGAX & 37 & HLA-J & 57 & CTSW & 77 & CD40 & 97 & RPLP0 \\
18 & CST3 & 38 & HLA-K & 58 & CTSX & 78 & CD83 & 98 & RPS18 \\
19 & HLA-B & 39 & HLA-L & 59 & CTSY & 79 & CD163 & 99 & RPS27A \\
20 & C1QA & 40 & HLA-M & 60 & CTSF & 80 & CD11B & 100 & RPL13A \\
\bottomrule
\end{tabular}
\caption{Top 100 genes ranked by statistical significance using Storey's FDR correction.}
\end{table}
The following Figures S1 and S2 display gene-gene interactions and ROC curves, respectively.

\begin{figure}[H]
\centering
\includegraphics[width=0.85\textwidth]{heatmap.png}
\caption{Heatmap of attention-weighted gene-gene interactions derived from GAT. Higher weights indicate stronger biological relevance}
\end{figure}

\begin{figure}[H]
\centering
\includegraphics[width=0.85\textwidth]{ROC.png}
\caption{ROC curves comparing the proposed framework with DIABLO, MOCAT, AMOGEL, and MOM-LIN across multi-omics datasets.}
\end{figure}
Figure S2 shows that our framework achieves superior classification performance compared to baseline models. The steep rise and high plateau of the ROC curve indicate strong sensitivity at low false positive rates, reflecting the model’s ability to reliably distinguish Alzheimer’s Disease from control samples.
Table S2 shows pathway enrichment analysis.
\begin{table}[H]
\centering
\scriptsize
\begin{tabular}{lll}
\toprule
\textbf{Dataset} & \textbf{Enriched Pathway} & \textbf{Adjusted p-value} \\
\midrule
ADNI & Neuroinflammation (TREM2–PLCG2 axis) & 1.2e-05 \\
ADNI & Lipid metabolism (APOE–CLU) & 3.4e-04 \\
ADNI & Tau pathology (MAPT–GRN) & 2.1e-03 \\
ROSMAP & Microglial activation (SPI1–CD33) & 9.8e-06 \\
ROSMAP & Amyloid processing (APP–PSEN1) & 4.5e-04 \\
ROSMAP & HLA-mediated immune signaling (HLA-B–CR1) & 1.7e-03 \\
\bottomrule
\end{tabular}
\caption{Pathway enrichment analysis using Reactome and KEGG databases. Results highlight key biological circuits implicated in Alzheimer's disease across ADNI and ROSMAP cohorts.}
\end{table}
\section*{Supplementary Discussion}

\subsection*{Functional Interpretation and Experimental Validation of Key Gene-Gene Interactions}

To enhance biological interpretability and translational relevance, we provide functional annotations and propose experimental validation strategies for the top gene-gene interactions identified by our framework. These insights are grounded in curated databases (Reactome, KEGG, GeneCards) and supported by recent experimental literature. Each interaction is contextualized by cell-type specificity and disease stage relevance, with recommendations for future in vivo and organoid-based validation.

\paragraph{TREM2–PLCG2 (Neuroinflammation Axis)}
TREM2 encodes a microglial receptor involved in phagocytosis and immune regulation, while PLCG2 encodes a downstream effector in lipid-mediated signaling. Their interaction is enriched in disease-associated microglia (DAM) during early-stage AD \citep{geneCardsTREM2, reactomeTREM2PLCG2, magno2021trem2Plcg2, keren2020dam}. CRISPR-Cas9 knockout of PLCG2 in TREM2-overexpressing microglial cultures reduced IL-6 secretion and impaired phagocytosis \citep{chang2023crisprMicroglia, Obst2021PLCG2TREM2}. Spatial transcriptomics confirmed regional co-expression in inflamed cortical areas \citep{zhou2020spatial}. Further validation using knock-in mouse models and spatial proteomics is recommended.

\paragraph{MAPT–GRN (Tau-Modulatory Axis)}
MAPT encodes tau, a key protein in neurofibrillary tangle formation, while GRN encodes progranulin, a lysosomal regulator. Their interaction links tau aggregation with lysosomal dysfunction \citep{geneCardsMAPT, reactomeMAPTGRN}. GRN knockdown in tau-overexpressing iPSC-derived neurons increased phosphorylated tau and reduced lysosomal markers. Co-immunoprecipitation confirmed physical association \citep{petkau2016grn, minami2022lysosome}. This axis is most active in excitatory neurons during mid-to-late AD stages \citep{mathys2019singlecell}. Future studies should employ 3D brain organoids and longitudinal imaging.

\paragraph{SPI1–CD33 (Microglial Regulation)}
SPI1 (PU.1) is a transcription factor regulating myeloid lineage commitment; CD33 is a microglial immune checkpoint receptor. SPI1 binds CD33 promoter regions, and its interference reduces CD33 expression while enhancing amyloid-beta phagocytosis \citep{geneCardsSPI1, reactomeSPI1CD33, Hansen2018CD33}. This axis transitions from homeostatic to activated microglia during AD progression. Inducible SPI1 knockdown in aged AD mouse models could clarify its temporal dynamics and therapeutic relevance.

\paragraph{APOE–CLU (Lipid Metabolism)}
APOE and CLU are apolipoproteins involved in cholesterol transport and amyloid clearance. Their co-expression in astrocytic subpopulations correlates with lipid dysregulation in AD brains \citep{geneCardsAPOE, reactomeAPOECLU, lau2020astrocyte}. Dual knockdown in astrocyte cultures altered lipid profiles and reduced amyloid uptake. Spatial lipid imaging and astrocyte-specific knockouts are needed to dissect their functional roles.

\paragraph{APP–PSEN1 (Amyloid Processing)}
APP encodes the amyloid precursor protein; PSEN1 is a γ-secretase subunit. Their interaction governs amyloid-beta production and Notch signaling \citep{reactomeAPPPSEN1, geneCardsAPP}. Co-immunoprecipitation and proximity ligation assays confirmed complex formation, and PSEN1 knockdown reduced amyloid-beta generation. This interaction is broadly active across neuronal subtypes, with enhanced activity in deep-layer pyramidal neurons \citep{zhang2015structure}. Validation via multiplexed proteomics and electrophysiology is suggested.

\paragraph{HLA-B–CR1 (Immune Signaling)}
HLA-B is a class I MHC molecule; CR1 is a complement receptor mediating immune complex clearance. Their interaction promotes antigen presentation and neuroinflammation \citep{reactomeHLABCR1, geneCardsCR1}. CR1 overexpression in HLA-B+ macrophages increased C3b binding and TNF-α secretion. Spatial proteomics revealed co-localization in AD-affected cortical regions, particularly in perivascular macrophages and infiltrating monocytes \citep{gate2020peripheral}. Blood-brain barrier organoids and single-cell cytokine profiling could further elucidate peripheral-central immune crosstalk.

These interactions exhibit distinct cell-type specificity—microglial (TREM2–PLCG2, SPI1–CD33), astrocytic (APOE–CLU), and neuronal (MAPT–GRN, APP–PSEN1)—and stage-dependent activity across AD progression. While current validations offer mechanistic insight, limitations remain due to reliance on 2D cultures and immortalized cell lines. Future efforts should incorporate 3D organoids, spatial transcriptomics, and inducible in vivo models. Expanding to multi-ethnic cohorts and integrating clinical metadata will be essential for translational deployment.















\begin{figure}[H]
\centering
\includegraphics[width=0.85\textwidth]{5.png} 
\caption{UMAP projection of latent embeddings from MOVE. AD and control samples form distinct clusters along the UMAP1 axis, indicating biologically coherent structure.}
\label{fig:umap_projection}
\end{figure}
Figure S3 illustrates UMAP projection of latent embeddings derived from the MOVE framework, revealing two well-separated molecular clusters corresponding to Alzheimer's Disease (AD) and control samples \citep{mcinnes2018umap}. The distinct spatial segregation along the UMAP1 axis suggests that the learned representations capture biologically coherent structure, enabling robust discrimination of disease states. This separation highlights the framework’s ability to encode disease-relevant signatures across modalities.

\subsection*{Data Availability and Ethics}

This study involves secondary analysis of publicly available datasets under approved data use agreements. No new human or animal experiments were conducted. ADNI data are accessible via the Alzheimer's Disease Neuroimaging Initiative (\url{https://adni.loni.usc.edu/}), and ROSMAP data are available through Synapse (\url{https://www.synapse.org/#!Synapse:syn3219045}). Access is subject to institutional approval and data governance policies and all procedures comply with the ethical guidelines of the respective data providers.

\section*{Numerical Code}

The complete Python implementation of our framework is openly available at \url{https://github.com/joonsungkang0223/MOBD}, including preprocessing scripts, model training modules, evaluation pipelines, and pretrained checkpoints. The repository follows FAIR principles and includes:

\begin{itemize}
  \item A comprehensive \texttt{README.md} with installation and usage instructions
  \item Modular scripts for data handling, training, and visualization
  \item Sample datasets and configuration files for replicating key results
  \item Environment specifications (\texttt{requirements.txt}) for reproducibility
  \item Pretrained models and output examples for benchmarking
\end{itemize}

Users can follow the documented pipeline to reproduce all results. For inquiries or contributions, please refer to the issue tracker and contact details provided in the repository.

















\bibliography{ref3}
\end{document}